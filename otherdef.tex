% ez valahol kellett
\DeclarePairedDelimiter\ceil{\lceil}{\rceil}
\DeclarePairedDelimiter\floor{\lfloor}{\rfloor}

% rövid alakok a hosszabb szavakra
\newcommand{\fv}[1]{%
függvény#1
}

\newcommand{\Fv}[1]{%
Függvény#1
}

\newcommand{\der}[1]{%
derivált#1
}

\newcommand{\Der}[1]{%
derivált#1
}

\newcommand{\de}[1]{%
differenciálegyenlet#1
}

\newcommand{\De}[1]{%
Differenciálegyenlet#1
}

\newcommand{\kef}[1]{%
kezdeti érték feladat#1
}

\newcommand{\Kef}[1]{%
Kezdeti érték feladat#1
}


\newcommand{\amgm}[1]{%
számtani-mértani közép#1
}
\newcommand{\Amgm}[1]{%
Számtani-mértani közép#1
}
\newcommand{\gmhm}[1]{%
mértani-harmonikus közép#1
}
\newcommand{\Gmhm}[1]{%
Mértani-harmonikus közép#1
}


\newcommand{\szigmon}[1]{%
szigorúan monoton#1
}
\newcommand{\Szigmon}[1]{%
Szigorúan monoton#1
}
\newcommand{\nov}[1]{%
növekvő#1
}
\newcommand{\Nov}[1]{%
Növekvő#1
}













% zárójelek stb.
\newcommand{\gzjel}[1]{%
{ \left( #1 \right) }
}
\newcommand{\Toligv}[2]{%
#1,\hdots ,#2
}


\newcommand{\GZJ}[1]{%
{ \left( #1 \right) }
}

\newcommand{\KZJ}[1]{%
{ \{ #1 \} }
}

\newcommand{\SZZJ}[1]{%
{ \left[ #1 \right] }
}

\newcommand{\Tolig}[2]{%
#1\hdots #2
}

\newcommand{\SZOR}[2]{%
#1\cdot\hdots\cdot #2
}

\newcommand{\mint}[2]{%
\int #1 \text{d}#2
}
