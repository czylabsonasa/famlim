
%usepackage es egyebek
\input{globdef.tex}

%zarojel, FA...
\input{locdef.tex}

% zárójelek; rövidítések \de=differenciálegyenlet
% ez valahol kellett
\DeclarePairedDelimiter\ceil{\lceil}{\rceil}
\DeclarePairedDelimiter\floor{\lfloor}{\rfloor}

% rövid alakok a hosszabb szavakra
\newcommand{\fv}[1]{%
függvény#1
}

\newcommand{\Fv}[1]{%
Függvény#1
}

\newcommand{\der}[1]{%
derivált#1
}

\newcommand{\Der}[1]{%
derivált#1
}

\newcommand{\de}[1]{%
differenciálegyenlet#1
}

\newcommand{\De}[1]{%
Differenciálegyenlet#1
}

\newcommand{\kef}[1]{%
kezdeti érték feladat#1
}

\newcommand{\Kef}[1]{%
Kezdeti érték feladat#1
}


\newcommand{\amgm}[1]{%
számtani-mértani közép#1
}
\newcommand{\Amgm}[1]{%
Számtani-mértani közép#1
}
\newcommand{\gmhm}[1]{%
mértani-harmonikus közép#1
}
\newcommand{\Gmhm}[1]{%
Mértani-harmonikus közép#1
}

\newcommand{\amqm}[1]{%
számtani-négyzetes közép#1
}
\newcommand{\Amqm}[1]{%
Számtani-négyzetes közép#1
}




\newcommand{\szigmon}[1]{%
szigorúan monoton#1
}
\newcommand{\Szigmon}[1]{%
Szigorúan monoton#1
}
\newcommand{\nov}[1]{%
növekvő#1
}
\newcommand{\Nov}[1]{%
Növekvő#1
}













% zárójelek stb.
\newcommand{\gzjel}[1]{%
{ \left( #1 \right) }
}
\newcommand{\Toligv}[2]{%
#1,\hdots ,#2
}


\newcommand{\GZJ}[1]{%
{ \left( #1 \right) }
}

\newcommand{\KZJ}[1]{%
{ \{ #1 \} }
}

\newcommand{\SZZJ}[1]{%
{ \left[ #1 \right] }
}

\newcommand{\Tolig}[2]{%
#1\hdots #2
}

\newcommand{\SZOR}[2]{%
#1\cdot\hdots\cdot #2
}

\newcommand{\mint}[2]{%
\int #1 \text{d}#2
}


% itt van a feladatok listája
\begin{document}\begin{spacing}{1.2}


   \section*{Nevezetes határértékek}

   \section*{Tartalom} \label{Tart}
         \nameref{e}\newline
         \nameref{nthroot}\newline
      \newpage
      \section*{az e-szám} \label{e}
      \section*{Tartalom-e} \label{Tarte}
         \nameref{e1}\newline
         \kek{
         \hfill{}\nameref{Tart}
         }\newpage
         \section*{Alap} \label{e1}
         \Fa{
            \MatTag{ebase}{
   a_n=\gzjel{ 1+\frac{1}{n}}^{n} \hspace{0.5cm} \nearrow\\
   b_n=\gzjel{ 1+\frac{1}{n}}^{n+1} \hspace{0.5cm} \searrow\\
   a_n < b_m\hspace{1cm} \forall n,m
}
Azaz:
\Mat{
   \exists \lim_{n\to \infty} \gzjel{ 1+\frac{1}{n}}^{n}=\mathrm{e}
}

         }
         \Mo{
         \nameref{e1Mo}
         \hfill\nameref{Tarte}
         }
         \newpage
         \section*{Alap-Mo} \label{e1Mo}
         \Mo{
            A \amgm{}miatt:
\szurkeM{
   1\gzjel{ 1+\frac{1}{n}}^{n}<
   \gzjel{\frac{1 + 1+\frac{1}{n}+\cdots+1+\frac{1}{n}}{n+1}}^{n+1}=
   \gzjel{ 1+\frac{1}{n+1}}^{n+1}
}
az $a_n=\gzjel{1+\frac{1}{n}}^{n}$ \szigmon{}nő.


A \gmhm{}miatt:
\szurkeM{
   1\gzjel{ 1+\frac{1}{n}}^{n+1}>
   \gzjel{\frac{n+2}{1 + 1-\frac{1}{n+1}+\cdots+1-\frac{1}{n+1}}}^{n+2}=\\
   =\gzjel{ 1+\frac{1}{n+1}}^{n+2}
}
az $b_n=\gzjel{1+\frac{1}{n}}^{n+1}$ \szigmon{}csökken.
Könnyen látható, hogy:
\szurkeM{
   a_n < b_m \hspace{1cm} \forall m,n \\
   b_n-a_n=\frac{a_n}{n}
}
Mindezek miatt a sorozatok konvergensek és a határértékeik egybeesnek. Ezt
a számot $e$-vel szokták jelölni.

         }
         \Fa{
         \nameref{e1}
         }
         \newpage
      \section*{n-edik gyök} \label{nthroot}
      \section*{Tartalom-nthroot} \label{Tartnthroot}
         \nameref{nthroot1}\newline
         \nameref{nthroot2}\newline
         \kek{
         \hfill{}\nameref{Tart}
         }\newpage
         \section*{Konstans} \label{nthroot1}
         \Fa{
            \szurkeM{
   a^{\frac{1}{n}} \to 1 \hspace{1cm} a\in\mathbb{R}
}

         }
         \Mo{
         \nameref{nthroot1Mo}
         \hfill\nameref{Tartnthroot}
         }
         \newpage
         \section*{Konstans-Mo} \label{nthroot1Mo}
         \Mo{
            Legyen $a>1$, ekkor valamely $a_n>0$ sorozattal $a^{\frac{1}{n}}=1+a_n$.
A következő megállapításokat tehetjük:
\szurkeM{
   a=\gzjel{1+a_n}^n \ge 1+na_n \hspace{1cm}(\text{Bernoulli})\\
   \frac{a-1}{n}\ge a_n \\
   1\le a^{\frac{1}{n}}\le 1+\frac{a-1}{n}\hspace{1cm}(\text{rendőr-elv})
}
$a<1$ esetén alkalmazzuk $\frac{1}{a}$-ra a fentieket.

         }
         \Fa{
         \nameref{nthroot1}
         }
         \newpage
         \section*{$n$} \label{nthroot2}
         \Fa{
            \Mat{
   \gzjel{ 1-\frac{1}{n}}^{n}\nearrow \frac{1}{\mathrm{e}}\\
   \gzjel{ 1-\frac{1}{n+1}}^{n}\searrow \frac{1}{\mathrm{e}}
}

         }
         \Mo{
         \nameref{nthroot2Mo}
         \hfill\nameref{Tartnthroot}
         }
         \newpage
         \section*{$n$-Mo} \label{nthroot2Mo}
         \Mo{
            A következő megállapításokat tehetjük:
\szurkeM{
   n=\gzjel{1+a_n}^n \ge 1+\frac{n(n-1)}{2}a^2_n \hspace{1cm}(\text{Binomiális})\\
   \sqrt{\frac{2}{n}}\ge a_n \\
   1\le n^{\frac{1}{n}}\le 1+\sqrt{\frac{2}{n}}\\\
   n^{\frac{1}{n}}\to 1 \hspace{1cm}(\text{rendőr-elv})
}

         }
         \Fa{
         \nameref{nthroot2}
         }
         \newpage

\end{spacing}
\end{document}

