
%usepackage es egyebek
\documentclass[17pt]{extarticle}


\usepackage[margin=1.5cm,nohead]{geometry}
\usepackage[utf8]{inputenc}
\usepackage[hungarian]{babel}


\usepackage[%hypertex,
   unicode=true,
   plainpages = false,
   pdfpagelabels,
   bookmarks=true,
   bookmarksnumbered=true,
   bookmarksopen=true,
   breaklinks=true,
   backref=false,
   colorlinks=true,
   linkcolor = blue,
   urlcolor  = blue,
   citecolor = red,
   anchorcolor = green,
   hyperindex = true,
   hyperfigures,
%   pdftex
]{hyperref}
\hypersetup{
   pdftitle={Feladatok},
   pdfauthor={Czylabson Asa},
   pdfsubject={Hold Föld Nap},
}


\usepackage{amsmath}
\usepackage{amsfonts}
\usepackage{amssymb}
\usepackage{graphicx}
\usepackage{type1cm}
\usepackage{setspace}
\usepackage{mathtools}



%\usepackage{tikz,lipsum,lmodern}
\usepackage[most]{tcolorbox}



\usepackage{matlab-prettifier}
\usepackage[T1]{fontenc}
\usepackage{listingsutf8}

\lstset{
   style              = Matlab-editor,
   basicstyle         = \mlttfamily,
   escapechar         = ",
   mlshowsectionrules = true,
   literate= {á}{{\'{a}}}1 {Á}{{\'{A}}}1 {é}{{\'{e}}}1 {É}{{\'{E}}}1 {í}{{\'{i}}}1 {Í}{{\'{I}}}1 {ó}{{\'{o}}}1 {Ó}{{\'{O}}}1  {ö}{{\"{o}}}1 {Ö}{{\"{O}}}1 {ő}{{\H{o}}}1 {Ő}{{\H{O}}}1 {ú}{{\'{u}}}1 {Ú}{{\'{U}}}1  {ü}{{\"{u}}}1 {Ü}{{\"{U}}}1 {ű}{{\H{u}}}1 {Ű}{{\H{U}}}1
}


%zarojel, FA...
\newcommand{\feher}[1]{%
\begin{tcolorbox}[colback=white]
#1
\end{tcolorbox}
}

\newcommand{\zold}[1]{%
\begin{tcolorbox}[colback=green!17]
#1
\end{tcolorbox}
}

\newcommand{\szurke}[1]{%
\begin{tcolorbox}[colback=gray!10!white]
#1
\end{tcolorbox}
}

\newcommand{\szurkeM}[1]{%
\begin{tcolorbox}[colback=gray!10!white, top=-5mm, bottom=3mm]
\begin{gather*}
#1
\end{gather*}
\end{tcolorbox}
}


\newcommand{\szurkeMB}[1]{%
\begin{tcolorbox}[colback=gray!10!white, top=-5mm, bottom=3mm]
\begin{gather*}
#1
\end{gather*}
\end{tcolorbox}
}



\newcommand{\sarga}[1]{%
\begin{tcolorbox}[colback=yellow!10!white]
#1
\end{tcolorbox}
}

\newcommand{\barna}[1]{%
\begin{tcolorbox}[colback=brown!20!white]
#1
\end{tcolorbox}
}

\newcommand{\kek}[1]{%
\begin{tcolorbox}[colback=blue!20!white]
#1
\end{tcolorbox}
}

\newcommand{\Fa}[1]{%
\begin{tcolorbox}[colback=brown!20!white]
#1
\end{tcolorbox}
}

\newcommand{\Fnew}[0]{%
\end{tcolorbox}
\begin{tcolorbox}[colback=brown!20!white]
}




\newcommand{\Mo}[1]{%
\begin{tcolorbox}[colback=green!17]
#1
\end{tcolorbox}
}


\newcommand{\Mnew}[0]{%
\end{tcolorbox}
\begin{tcolorbox}[colback=green!17]
}


\newcommand{\egy}[1]{%
\Fa{\input{#1Fa}}
\Mo{\input{#1Mo}}
}

\newcommand{\tg}[0]{%
\text{tg}
}
\newcommand{\ctg}[0]{%
\text{ctg}
}





% zárójelek; rövidítések \de=differenciálegyenlet
% ez valahol kellett
\DeclarePairedDelimiter\ceil{\lceil}{\rceil}
\DeclarePairedDelimiter\floor{\lfloor}{\rfloor}

% rövid alakok a hosszabb szavakra
\newcommand{\fv}[1]{%
függvény#1
}

\newcommand{\Fv}[1]{%
Függvény#1
}

\newcommand{\der}[1]{%
derivált#1
}

\newcommand{\Der}[1]{%
derivált#1
}

\newcommand{\de}[1]{%
differenciálegyenlet#1
}

\newcommand{\De}[1]{%
Differenciálegyenlet#1
}

\newcommand{\kef}[1]{%
kezdeti érték feladat#1
}

\newcommand{\Kef}[1]{%
Kezdeti érték feladat#1
}


\newcommand{\amgm}[1]{%
számtani-mértani közép#1
}
\newcommand{\Amgm}[1]{%
Számtani-mértani közép#1
}
\newcommand{\gmhm}[1]{%
mértani-harmonikus közép#1
}
\newcommand{\Gmhm}[1]{%
Mértani-harmonikus közép#1
}


\newcommand{\szigmon}[1]{%
szigorúan monoton#1
}
\newcommand{\Szigmon}[1]{%
Szigorúan monoton#1
}
\newcommand{\nov}[1]{%
növekvő#1
}
\newcommand{\Nov}[1]{%
Növekvő#1
}













% zárójelek stb.
\newcommand{\gzjel}[1]{%
{ \left( #1 \right) }
}
\newcommand{\Toligv}[2]{%
#1,\hdots ,#2
}


\newcommand{\GZJ}[1]{%
{ \left( #1 \right) }
}

\newcommand{\KZJ}[1]{%
{ \{ #1 \} }
}

\newcommand{\SZZJ}[1]{%
{ \left[ #1 \right] }
}

\newcommand{\Tolig}[2]{%
#1\hdots #2
}

\newcommand{\SZOR}[2]{%
#1\cdot\hdots\cdot #2
}

\newcommand{\mint}[2]{%
\int #1 \text{d}#2
}


% itt van a feladatok listája
\begin{document}\begin{spacing}{1.2}


   \section*{Nevezetes határértékek}

   \section*{Tartalom} \label{Tart}
         \nameref{e}\newline
         \nameref{nthroot}\newline
      \newpage
      \section*{az e-szám} \label{e}
      \section*{Tartalom-e} \label{Tarte}
         \nameref{e1}\newline
         \kek{
         \hfill{}\nameref{Tart}
         }\newpage
         \section*{Alap} \label{e1}
         \Fa{
            \Mat{
   \exists \lim_{n\to \infty} \gzjel{ 1+\frac{1}{n}}^{n}
}

         }
         \Mo{
         \nameref{e1Mo}
         \hfill\nameref{Tarte}
         }
         \newpage
         \section*{Alap-Mo} \label{e1Mo}
         \Mo{
            A \amgm{}miatt:
\Mat{
   1\gzjel{ 1+\frac{1}{n}}^{n}<
   \gzjel{\frac{1 + 1+\frac{1}{n}+\cdots+1+\frac{1}{n}}{n+1}}^{n+1}=
   \gzjel{ 1+\frac{1}{n+1}}^{n+1}
}
az $a_n=\gzjel{1+\frac{1}{n}}^{n}$ \szigmon{}nő.


A \gmhm{}miatt:
\Mat{
   1\gzjel{ 1+\frac{1}{n}}^{n+1}>
   \gzjel{\frac{n+2}{1 + 1-\frac{1}{n+1}+\cdots+1-\frac{1}{n+1}}}^{n+2}=\\
   =\gzjel{ 1+\frac{1}{n+1}}^{n+2}
}
az $b_n=\gzjel{1+\frac{1}{n}}^{n+1}$ \szigmon{}csökken.
Könnyen látható, hogy:
\Mat{
   a_n < b_m \hspace{1cm} \forall m,n \\
   b_n-a_n=\frac{a_n}{n}
}
Mindezek miatt a sorozatok konvergensek és a határértékeik egybeesnek. Ezt
a számot e-vel szokták jelölni.

         }
         \Fa{
         \nameref{e1}
         }
         \newpage
      \section*{n-edik gyök} \label{nthroot}
      \section*{Tartalom-nthroot} \label{Tartnthroot}
         \nameref{nthroot1}\newline
         \nameref{nthroot2}\newline
         \kek{
         \hfill{}\nameref{Tart}
         }\newpage
         \section*{Konstans} \label{nthroot1}
         \Fa{
            \szurkeM{
   a^{\frac{1}{n}} \to 1 \hspace{1cm} a\in\mathbb{R}
}

         }
         \Mo{
         \nameref{nthroot1Mo}
         \hfill\nameref{Tartnthroot}
         }
         \newpage
         \section*{Konstans-Mo} \label{nthroot1Mo}
         \Mo{
            Legyen $a>1$, ekkor valamely $a_n>0$ sorozattal $a^{\frac{1}{n}}=1+a_n$.
A következő megállapításokat tehetjük:
\szurkeM{
   a=\gzjel{1+a_n}^n \ge 1+na_n \hspace{1cm}(\text{Bernoulli})\\
   \frac{a-1}{n}\ge a_n \\
   1\le a^{\frac{1}{n}}\le 1+\frac{a-1}{n}\hspace{1cm}(\text{rendőr-elv})
}
$a<1$ esetén alkalmazzuk $\frac{1}{a}$-ra a fentieket.

         }
         \Fa{
         \nameref{nthroot1}
         }
         \newpage
         \section*{$n$} \label{nthroot2}
         \Fa{
            \szurkeM{
   n^{\frac{1}{n}} \to 1
}

         }
         \Mo{
         \nameref{nthroot2Mo}
         \hfill\nameref{Tartnthroot}
         }
         \newpage
         \section*{$n$-Mo} \label{nthroot2Mo}
         \Mo{
            A következő megállapításokat tehetjük:
\szurkeM{
   n=\gzjel{1+a_n}^n \ge 1+\frac{n(n-1)}{2}a^2_n \hspace{1cm}(\text{Binomiális})\\
   \sqrt{\frac{2}{n}}\ge a_n \\
   1\le n^{\frac{1}{n}}\le 1+\sqrt{\frac{2}{n}}\hspace{1cm}(\text{rendőr-elv})
}

         }
         \Fa{
         \nameref{nthroot2}
         }
         \newpage

\end{spacing}
\end{document}

